\documentclass[10pt]{article}
%-------Genéricos---------%
\setlength{\hoffset}{9.6mm}
\setlength{\voffset}{-10 mm}
\setlength{\oddsidemargin}{0 mm}
\setlength{\topmargin}{0 mm}
\setlength{\headheight}{10mm}
\setlength{\headsep}{12pt}
\setlength{\textwidth}{150 mm}
\setlength{\textheight}{235 mm}
\setlength{\marginparsep}{0 mm}
\setlength{\footskip}{ 10mm}
\setlength{\marginparpush}{5pt}
\setlength{\marginparwidth}{25mm}
%-------Tipo de letra arial---------%
\renewcommand{\rmdefault}{phv} % Arial
\renewcommand{\sfdefault}{phv} % Arial
%-------Paquete para figuras---------%
\usepackage{float}
\usepackage{graphicx}
%-------Paquete para escritura en español---------%
\usepackage[T1]{fontenc}
\usepackage[spanish]{babel}
%----Paquetes para los titulos de graficos y tablas
\usepackage{caption}
\captionsetup{font=normal,labelfont=bf}
\usepackage{booktabs}
%-------Interlineado-------------%
\usepackage{setspace}
\doublespacing

\begin{document}

%------------Portada--------------%

%------------Portada--------------%
\thispagestyle{empty} % no imprimir ni número, ni cabecera ni pie de página
\vspace*{5mm}
\begin{center}
\begin{figure}[h]
	\centering
		\includegraphics[height=25mm, width = 65.64mm]{Imagenes/escudouna.pdf}
	\label{fig:escudouna}
\end{figure}
\vspace{10pt}
\begin{spacing}{1.25}
	\Large ESCUELA TÉCNICA SUPERIOR DE INGENIERÍA DE TELECOMUNICACIÓN\\
\end{spacing}
\vspace{18pt}
\Large GRADO EN SISTEMAS DE TELECOMUNICACIÓN\\
\vspace{48pt}
\textbf{TRABAJO FIN DE GRADO}
\vspace{48pt}
\begin{spacing}{1.25}
	\LARGE ESTABILIZACIÓN DE UN DRONE USANDO FPGAs LIBRES\\
	\vspace{48pt}
\end{spacing}

Autor: Eloy Navarro Morales\\
Tutor: José María Cañas Plaza\\
Cotutor: Juan Ordoñez Cerezo\\

\vspace{24pt} % era 35mm

\large Curso académico 2018/2019
\end{center}
%------------END Portada--------------%



%------------Indice--------------%
\tableofcontents


%------------Estructura--------------%
\hfill \break
\section{Introducción}
	\subsection{Aplicaciones para drones}
	\subsection{Sistemas básicos del drone}
		\subsubsection{Hardware}
			\paragraph{Comunicaciones}
			\paragraph{Electrónica de Control}
			\paragraph{Drivers}
		\subsubsection{Software}
	\subsection{FPGA}
		\subsubsection{Concepto}
		\subsubsection{Aplicaciones}
		\subsubsection{FPGAs Libres}
	\subsection{Sistemas de control}
		\subsubsection{Bucle abierto}
		\subsubsection{Bucle cerrado}
		
\section{Objetivos}
	\subsection{Objetivo principal}
	Controlar un drone de bajo coste usando un PC y FPGAs libres.
	\subsection{Requisitos}
		\subsubsection{Estabilización de un drone de bajo coste}
		\subsubsection{Diseño de un sistema de control basado en FPGAs libres}
		\subsubsection{Diseño de electrónica periférica para comunicaciones entre PC, Drone y Electrónica de control}
		\subsubsection{Implementación de librería de control del drone para PC}
			\paragraph{Parámetros reconfigurables}
			\paragraph{Control de Posición}
			\paragraph{Control de Trayectoria}
		\subsubsection{Software}
	\subsection{Metodología}
	Diseño, Implementación, test, análisis de resultados, informes y feedback vía mail y conferencia.
	\subsection{Plan de trabajo}
	Se inicia el proyecto con un control teledirigido directo del drone.\\
	Se procede a diseñar todo el sistema para un control en bucle abierto.\\
	Se continúa con la electrónica y software necesarios para cerrar un eje.\\
	Se amplía el sistema para controlar los 3 ejes principales.
			
\section{Arquitectura del sistema}
	\subsection{Estación de tierra}
		\subsubsection{Descripción}
		\subsubsection{Objetivo}
		\subsubsection{Subsistemas}
			\paragraph{Radio}
			\paragraph{Procesador principal FPGA}
			\paragraph{Procesador secundario ATMEL}
		\subsubsection{Interfaces externos}
			\paragraph{Enlaces Radio}
			\paragraph{USB}
			\paragraph{Programación}
		\subsubsection{Interfaces internos}
			\paragraph{SPI}
			\paragraph{RS232}
	\subsection{Sistemas embarcados}
		\subsubsection{Descripción}
		\subsubsection{Objetivo}
		\subsubsection{Subsistemas}
			\paragraph{Radio}
			\paragraph{Procesador}
			\paragraph{Sensores}
		\subsubsection{Interfaces externos}
			\paragraph{Radio }
			\paragraph{Programación}
		\subsubsection{Interfaces internos}
			\paragraph{I2C}
			\paragraph{SPI}
			
\section{Algoritmos de control}
	\subsection{Controles de bucle abierto}
		\subsubsection{Directo}
		\subsubsection{Pre-énfasis}
	\subsection{Controles de bucle cerrado}
		\subsubsection{PID en plano vertical retroalimentado}
		\subsubsection{PIDs en plano horizontal retroalimentados}
		

\section{Experimentos}
	\subsection{Eachine E010}
	\subsection{Syma X5C en Bucle abierto}
	
	
\section{Trabajo futuro}
	\subsection{Mejoras hardware}
	\subsection{Mejoras software}
	\subsection{Cambios a mejor en la arquitectura del sistema}
	
	
\section{7.	Conclusiones}

\section{8.	Bibliografía}

\end{document}